% Ementa: Sistemas Paralelos - Documento de 1 página
\documentclass[11pt,a4paper]{article}
\usepackage[utf8]{inputenc}
\usepackage[T1]{fontenc}
\usepackage[brazil]{babel}
\usepackage[margin=1.5cm]{geometry}
\usepackage{parskip}
\usepackage{enumitem}
\usepackage{titlesec}
\usepackage{microtype}
\renewcommand{\familydefault}{\sfdefault}
\titleformat{\section}{\large\bfseries}{}{0em}{}
\begin{document}
\thispagestyle{empty}
\begin{center}
  {\LARGE \textbf{Sistemas Paralelos -- Ementa (Resumida)}}\\[6pt]
  {\small Curso: (nome) \quad Semestre: ( ) \quad Carga hor\'aria: ( )}
\end{center}
\vspace{6pt}
\section*{Ementa}
Concorr\^encia, necessidade e implementa\c{c}\~ao; Controle de processos e threads; T\'ecnicas de otimiza\c{c}\~ao; An\'alise de desempenho; Paraleliza\c{c}\~ao fork-join; OpenMP; MPI; CUDA; Nuvem.

\section*{Objetivos}
Familiarizar o aluno com os conceitos e termos b\'asicos de sistemas paralelos, implementa\c{c}\~ao e uso de concorr\^encia; apresentar tipos de arquitetura; descrever suporte para programar tais sistemas e apresentar aplica\c{c}\~oes.

\section*{Conte\'udo Program\'atico}
\begin{itemize}[leftmargin=*,noitemsep,topsep=3pt]
  \item \textbf{Conceitos b\'asicos}: processos, threads, interrup\c{c}\oes, escalonamento.
  \item \textbf{Problemas de programa\c{c}\~ao concorrente}: deadlock, aloca\c{c}\ao de recursos.
  \item \textbf{Leitura/escrita concorrente}: exclus\~ao m\'utua, consenso.
  \item \textbf{Program. concorrente em UNIX}: sem\'aforos, mutexes e monitores.
  \item \textbf{Otimiza\c{c}\~ao sequencial}: uso eficiente da mem\'oria, unit stride, blocking.
  \item \textbf{Instru\c{c}\~oes vetoriais e superscalares}: op\c{c}\oes de otimiza\c{c}\~ao.
  \item \textbf{Profiling e modelagem de desempenho}.
  \item \textbf{Controle de processos e paraleliza\c{c}\~ao fork-join}.
  \item \textbf{Mem\'oria compartilhada e OpenMP} (introdu\c{c}\~ao).
  \item \textbf{Mem\'oria distribu\'ida e MPI}.
  \item \textbf{Program. em GPUs e CUDA} e tecnologias emergentes.
  \item \textbf{Computa\c{c}\~ao paralela na nuvem}.
\end{itemize}

\vfill
{\footnotesize Documento gerado: ementa compacta para um p\'agina. Ajuste cabe\c{c}alho/rodap\'e e carga hor\'aria conforme necessidade.}
\end{document}
